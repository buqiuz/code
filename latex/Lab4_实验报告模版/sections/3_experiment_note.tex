\section{实验过程记录}

\subsection{问题 1: 触发器的使能端问题}
问题描述:实验中 Controller 模块用到的流水线寄存器不含有使能端,datapath 模块使用的流
水线寄存器含有使能端,刚开始实验时混淆。

解决方案:Controller 模块使用的流水线寄存器为 floprc,即含有 rst 和 clear 信号的触发器,因
为此处流水线寄存器的功能只是在时钟信号到来时将数据传入或接受输入的控制信号,随时有
效,无需使能端;datapath 使用的流水线寄存器存储的是数据信号,只有在使能端有效的情况下才
输入输出。
\subsection{问题 2: 触发器的使能端问题}
问题描述:连接 datapath 时,判断需要分支时直接将当前 PC 值与 branch 字节地址相加。但
实际上此时的 PC 已经执行了 PC+4,存在偏移。

解决方案:在计算 branch 分支地址时将 PC-4 再传入。
\subsection{问题 3: 触发器的使能端问题}
问题描述:流水线寄存器的 rst 和 clear 看似作用相同,都是清空流水线实则不同

解决方案:。rst 相当于整个电路的开关,置 1 时整个电路归零,有可能是异常或流水线冲突;
clear 是较为简单的流水线清空,有可能是人为清空流水线结束 CPU 或者流水线地址跳转却已经
执行跳转前代码。因此需将 rst 和 clear 分开。